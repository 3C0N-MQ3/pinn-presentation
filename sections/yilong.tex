\begin{frame}{Application: PINN in Finance}
    Noguer i Alonso, Miquel and Antolin Camarena, Julian, Physics-Informed Neural Networks (PINNs) in Finance (October 10, 2023). Available at SSRN: \url{https://ssrn.com/abstract=4598180} or \url{http://dx.doi.org/10.2139/ssrn.4598180}
\end{frame}

\begin{frame}{Application: PINN in Finance}
    \begin{block}{Motivation I}
        In the study of finance, Partial Differential Equations (PDEs) and Stochastic Differential Equations (SDEs) are widely applied to model phenomena.  
    \end{block}

    \begin{block}{Motivation II}
        Physics-Informed Neural Networks (PINNs) provide a promising method to solve PDEs by embedding the physical rules into the architecture and training process. 
    \end{block}
\end{frame}

\begin{frame}{Dynamic Systems in Finance}
    \begin{block}{The Black-Scholes Model}
        \begin{equation}
            \frac{\partial V}{\partial t} + \frac{1}{2} \sigma^2 S^2 \frac{\partial^2 V}{\partial S^2} + r S \frac{\partial V}{\partial S} - r V = 0
        \end{equation}
    \end{block}

    \begin{block}{The Physical Part of Loss Function}
        \begin{equation}
            L_{\text{physics}} = \left| \frac{\partial f_{\theta}}{\partial t} + \frac{1}{2} \sigma^2 S^2 \frac{\partial^2 f_{\theta}}{\partial S^2} + r S \frac{\partial f_{\theta}}{\partial S} - r f_{\theta} \right|^2
        \end{equation}
    \end{block}
\end{frame}

\begin{frame}{Dynamic Systems in Finance}
    \begin{block}{The Heston Model}
        \begin{align}
            dS_t &= r S_t \, dt + \sqrt{v_t} S_t \, dW_t^S \\
            dv_t &= \kappa (\theta - v_t) \, dt + \xi \sqrt{v_t} \, dW_t^v
        \end{align}
        where the Feller condition:
        \begin{equation}
            2 \kappa \theta > \xi^2
        \end{equation}
    \end{block}

    \begin{block}{The Physical Part of Loss Function}
        \begin{equation}
            L_{\text{physics}} = \left| \frac{d f_{\theta}^S}{dt} - r f_{\theta}^S - \sqrt{f_{\theta}^v} f_{\theta}^S \right|^2 + \left| \frac{d f_{\theta}^v}{dt} - \kappa (\theta - f_{\theta}^v) - \xi \sqrt{f_{\theta}^v} \right|^2
        \end{equation}
    \end{block}
\end{frame}

\begin{frame}{PINN Approach with The Heston Model}
\framesubtitle{Experiment Set Up}
    \begin{enumerate}
        \item 10,000 trajectories of simulated asset price-volatility pairs.
        \item 252 days.
        \item \(\rho = 0.5\), \(\mu = 0.01\), \(\sigma_S = 2.0\), \(\xi = 0.1\), \(\theta = 0.2\), \(\kappa = 0.075\)
        \item Repeated 100 times
    \end{enumerate}
\end{frame}

\begin{frame}{Experiment Results}
    \begin{figure}[H]
        \centering
        \includegraphics[width=0.8\textwidth]{img/png3.png}
        \caption{Log-loss curves of training and validation losses.}
        \label{fig:log_loss}
    \end{figure}
\end{frame}

\begin{frame}{Experiment Results}
    \begin{figure}[H]
        \centering
        \includegraphics[width=0.6\textwidth]{img/figure2.png}
        \caption{Numerical results of Heston PINN. Top panel: Stock price over a trading year in blue, with the mean PINN prediction in black and the confidence bands at one daily standard deviation. Bottom panel: Volatility of the stock price over a trading year in blue, with the mean PINN prediction in black and the confidence bands at one daily standard deviation.}
        \label{fig:heston_results}
    \end{figure}
\end{frame}

\begin{frame}{Implication and Limitation}
    \begin{enumerate}
        \item PINNs train the Heston model adequately.
        \item The estimated mean is largely within one daily standard deviation of the ground truth, indicating that the Heston model is being learned relatively well.
        \item PINNs not only provide accurate predictions but also ensure that these predictions are consistent with known financial rules and structures. 
        \item The experiment conducted with Heston here is not feasible in reality.
    \end{enumerate}
\end{frame}

\begin{frame}{Implication and Limitation}
    Is it possible to apply this approach to economic studies like macroeconomics?

    Although there is no way to find a parallel universe to test our economic models, we may consider applying this theory to countries whose characteristics converge (for example, OECD countries).

    Xavier X. Sala-i-Martin. (1996). The Classical Approach to Convergence Analysis. The Economic Journal, 106(437), 1019–1036. \url{https://doi.org/10.2307/2235375}
\end{frame}