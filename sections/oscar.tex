\begin{frame}{Insight into Challenges \& Future Direction}
 
    \begin{block}{Challenge I: Sensitivity}
    
        One of the primary theoretical challenges is understanding the convergence properties of PINNs. Selecting appropriate hyperparameters is often a trial-and-error process that can be time-consuming and computationally expensive. 

    \end{block}

    \begin{block}{Challenge II: Stability \& Convergence}
    
        The optimization landscape of PINNs is typically non-convex, making it challenging to find global minima.  

    \end{block}
\end{frame}
\begin{frame}
 
    \begin{block}{Challenge III: Scalability and Computational Demands}
    
        PINNs need to be more efficient in handling high-dimensional and multi-scale problems.

    \end{block}

    \begin{block}{Challenge IV: Interpretability}
    
        Real-world data is often noisy and incomplete, which can negatively impact the performance of PINNs. 

    \end{block}
\end{frame}
\begin{frame}{Future Direction}
 
    \begin{block}{High-Dimensional and Multi-Scale Problems}
    
        Addressing the challenges of high-dimensional and multi-scale problems by integrating techniques such as dimensionality reduction, sparse modeling, and hierarchical decomposition. This will broaden the applicability of PINNs to more complex scientific and engineering problems. 

    \end{block}

    \begin{block}{Hybrid Models}
    
        Combining PINNs with traditional numerical methods or other machine learning approaches to leverage the strengths of each. Hybrid models can offer improved accuracy and computational efficiency for solving complex PDEs. 

    \end{block}
\end{frame}
\begin{frame}
 
    \begin{block}{Advancement in Software Tools}
    
        Develop comprehensive software tools and libraries that make it easier for researchers and practitioners to implement, train, and deploy PINNs. These tools should include intuitive interfaces, detailed documentation, and support for various PDEs and boundary conditions.

    \end{block}

    \begin{block}{Transfer Learning and Domain Adaptation}
    
        Tailoring PINNs for specific application domains, such as biomedical engineering, climate modeling, and financial forecasting. This involves incorporating domain-specific knowledge and constraints into the PINN framework.

    \end{block}
\end{frame}
\begin{frame}{Conclusion}
\begin{block}{Long-Term Utility of PINNs}
        \emph{The future direction of PINN research is focused on addressing current challenges, enhancing theoretical understanding, improving computational efficiency, and broadening their applicability through customization and educational initiatives. Given these efforts, PINNs have the potential to revolutionize the way complex differential equations are solved, leading to significant advancements in various scientific and engineering disciplines.}
    \end{block}
\end{frame}